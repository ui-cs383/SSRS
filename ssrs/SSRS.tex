% This file was converted to LaTeX by Writer2LaTeX ver. 1.0.2
% see http://writer2latex.sourceforge.net for more info
\documentclass[twoside,letterpaper]{article}

\usepackage[T1]{fontenc}
\usepackage[english]{babel}
\usepackage{amsmath}
\usepackage{amssymb,amsfonts,textcomp}
\usepackage{color}
\usepackage{array}
\usepackage{supertabular}
\usepackage{hhline}
\usepackage{hyperref}
\hypersetup{pdftex, colorlinks=true, linkcolor=black, citecolor=black, filecolor=black, urlcolor=black}
\usepackage[pdftex]{graphicx}


% Outline numbering
\setcounter{secnumdepth}{5}
%\renewcommand\thesection{\arabic{section}}
%\renewcommand\thesubsection{\arabic{section}.\arabic{subsection}}
%\renewcommand\thesubsubsection{\arabic{section}.\arabic{subsection}.\arabic{subsubsection}}
%\renewcommand\theparagraph{\arabic{section}.\arabic{subsection}.\arabic{subsubsection}.\arabic{paragraph}}
%\renewcommand\thesubparagraph{\arabic{section}.\arabic{subsection}.\arabic{subsubsection}.\arabic{paragraph}.\arabic{subparagraph}}
\makeatletter
\newcommand\arraybslash{\let\\\@arraycr}
\makeatother
% Page layout (geometry)
\setlength\voffset{-1in}
\setlength\hoffset{-1in}
\setlength\topmargin{0.5in}
\setlength\oddsidemargin{1in}
\setlength\evensidemargin{1in}
\setlength\textheight{8.25in}
\setlength\textwidth{6.5in}
\setlength\footskip{0.5in}
\setlength\headheight{0.5in}
\setlength\headsep{0.5in}
% Footnote rule
\setlength{\skip\footins}{0.0469in}
\renewcommand\footnoterule{\vspace*{-0.0071in}\setlength\leftskip{0pt}\setlength\rightskip{0pt plus 1fil}\noindent\textcolor{black}{\rule{0.25\columnwidth}{0.0071in}}\vspace*{0.0398in}}
% Pages styles
\makeatletter
\newcommand\ps@Standard{
  \renewcommand\@oddhead{\selectlanguage{english}\rmfamily\color{black} University of Idaho CS Department Instructional Use\hfill \hfill NOT FOR RELEASE}
  \renewcommand\@evenhead{\@oddhead}
  \renewcommand\@oddfoot{\foreignlanguage{english}{\textcolor{black}{SSRS Page }}\foreignlanguage{english}{\textcolor{black}{\thepage{}}}}
  \renewcommand\@evenfoot{\@oddfoot}
  \renewcommand\thepage{\arabic{page}}
}
\newcommand\ps@Convertviii{
  \renewcommand\@oddhead{}
  \renewcommand\@evenhead{\@oddhead}
  \renewcommand\@oddfoot{}
  \renewcommand\@evenfoot{\@oddfoot}
  \renewcommand\thepage{\arabic{page}}
}
\newcommand\ps@Convertvii{
  \renewcommand\@oddhead{}
  \renewcommand\@evenhead{\@oddhead}
  \renewcommand\@oddfoot{}
  \renewcommand\@evenfoot{\@oddfoot}
  \renewcommand\thepage{\arabic{page}}
}
\newcommand\ps@Convertvi{
  \renewcommand\@oddhead{}
  \renewcommand\@evenhead{\@oddhead}
  \renewcommand\@oddfoot{}
  \renewcommand\@evenfoot{\@oddfoot}
  \renewcommand\thepage{\arabic{page}}
}
\newcommand\ps@Convertv{
  \renewcommand\@oddhead{}
  \renewcommand\@evenhead{\@oddhead}
  \renewcommand\@oddfoot{}
  \renewcommand\@evenfoot{\@oddfoot}
  \renewcommand\thepage{\arabic{page}}
}
\newcommand\ps@Convertiv{
  \renewcommand\@oddhead{}
  \renewcommand\@evenhead{\@oddhead}
  \renewcommand\@oddfoot{}
  \renewcommand\@evenfoot{\@oddfoot}
  \renewcommand\thepage{\arabic{page}}
}
\newcommand\ps@Convertii{
  \renewcommand\@oddhead{}
  \renewcommand\@evenhead{\@oddhead}
  \renewcommand\@oddfoot{}
  \renewcommand\@evenfoot{\@oddfoot}
  \renewcommand\thepage{\arabic{page}}
}
\newcommand\ps@FirstPage{
  \renewcommand\@oddhead{}
  \renewcommand\@evenhead{\@oddhead}
  \renewcommand\@oddfoot{}
  \renewcommand\@evenfoot{\@oddfoot}
  \renewcommand\thepage{\arabic{page}}
}
\makeatother
\pagestyle{Standard}
\setlength\tabcolsep{1mm}
\renewcommand\arraystretch{1.3}
% footnotes configuration
\makeatletter
\renewcommand\thefootnote{\arabic{footnote}}
\makeatother
\title{SYSTEMS AND SOFTWARE REQUIREMENTS SPECIFICATION (SSRS) TEMPLATE}
\author{Clinton Jeffery}
\date{2010-11-18T11:33:37.30}


% % % % % % % % % % % % % % %
% Place my packages here    %
% % % % % % % % % % % % % % %

% New tables
\usepackage[utf8]{inputenc}
\usepackage{tabularx}
\usepackage{ragged2e}
\usepackage{booktabs}
\usepackage{caption}
\newcolumntype{C}[1]{>{\Centering}p{#1}}
\renewcommand\tabularxcolumn[1]{C{#1}}
%\usepackage{chngpage}

% Appendices
%\usepackage[titletoc,toc]{appendix}

% Use verbatim to allow block comments.
\usepackage{verbatim}
\usepackage{usecases}
%\usepackage{fullpage}
\usepackage{listings}
\usepackage{url}
\usepackage[parfill]{parskip}

% Automatically clearpage after every section
\usepackage{titlesec}
\newcommand\sectionbreak{\ifnum\value{section}>0\clearpage\fi}



\begin{document}

% Removing rule spacing for tables
\setlength{\aboverulesep}{0pt}
\setlength{\belowrulesep}{0pt}
\setlength{\extrarowheight}{.75ex}

% % % % % % % % % % % % %
% Cover Page            %
% % % % % % % % % % % % %
\begin{minipage}{\linewidth}
\centering
\textsc{
	\textbf{
		Systems and Software Requirements Specification (SSRS) 	\\
							For								   	\\
							\vspace{1em}
						{\Large Freedom in the Star System}		\\
}}
\end{minipage}

\vspace{2em}

\begin{figure}[h]
\centering
\includegraphics[scale=0.8]{./images/fitss.png}
\end{figure}

\begin{minipage}{\linewidth}
\centering
Version 0.2\\
17 December 2013\\

\vspace{2em}

Prepared for:\\
Dr. Clinton Jeffrey\\
\url{jeffreyc@uidaho.edu}\\
JEB 230, University of Idaho

\vspace{2em}

Prepared by:\\
Chris Waltrip\\
University of Idaho\\
Moscow, ID 83844-1010

\vspace{2em}

Material from:\\
CS383 Python Team\\
et. al. (Will eventually put all team member names in)\\
University of Idaho\\
Moscow, ID 83844-1010
\end{minipage}


\clearpage
% % % % % % % % % % % % % %
% Record of Changes Page  %
% % % % % % % % % % % % % %
\begin{minipage}{\linewidth}
\centering
\textsc{
	\textbf{Freedom in the Star System SSRS}}

\vspace{2em}

\textsc{
	\textbf{Record of Changes}}
	
\end{minipage}

\begin{minipage}{\linewidth}
% Table
% Create column type for ``Brief Description''
\newcolumntype{P}{>{\centering\arraybackslash}p{\dimexpr.25\linewidth-2\tabcolsep}}
\centering
%\captionof*{table}{\textsc{\textbf{Record of Changes}}}
\begin{tabularx}{\textwidth}{XXXXPXX}\toprule[1.5pt] % Description column is multiline

% Header
\bf Change Number & \bf Change Date & \bf Change Location & \bf A\newline M\newline D & \bf Brief \newline Description & \bf Approved By \newline (Initials) & \bf Date Approved \\ \midrule[1.0pt]

% Changes
1 & 12/5/13 & Section 1 & A & Initial copy, filled in details & --- & ---\\
2 & 12/17/13 & Section 3.2 & A & Added use cases & --- & ---\\
3 & 12/17/13 & All Tables & M & Changed table formats & --- & ---\\
4 & 12/17/13 & Preamble & M & Converted to minitables & --- & ---\\
5 & 12/17/13 & Table of Contents & M & Slight design changes (uniformity) & --- & ---\\
6 & 12/17/13 & Section 3.2 & M & Overhauled Use Case style & --- & ---\\
7 & 12/19/13 & Section 2 & A & Wrote most of Section 2 & --- & ---\\
8 & 12/19/13 & Section 3 & A & Wrote remaining portions of Section 3 except for use case diagrams & --- & ---\\

% Footer
\bottomrule[1.5pt]
\end{tabularx}\par

% Legend
\bigskip
\raggedleft % Align legend to the right
\begin{tabular}{c l}
\multicolumn{2}{c}{\textsc{Legend}} \\ \midrule[0.5pt]
\textsc{\textbf{A}}   & Added\\
\textsc{\textbf{M}}   & Modified\\
\textsc{\textbf{D}}	  & Deleted\\
\end{tabular}
\end{minipage}

\clearpage
% % % % % % % % % % % % % %
% Table of Contents Page  %
% % % % % % % % % % % % % %
\begin{minipage}{\linewidth}
\centering
\textsc{
	\textbf{Freedom in the Star System SSRS}}

\vspace{2em}

%\textsc{
%	\textbf{{\LARGE Table of Contents}}}

%\vspace{2em}

\end{minipage}

% Set up TOC
% Show lots of depth
\setcounter{tocdepth}{9}
% Center, bold and small caps ``Table of Contents''
\renewcommand\contentsname{\centering\textsc{\textbf{Table of Contents}}}
\tableofcontents
%\clearpage\clearpage %\setcounter{page}{1}\pagestyle{Convertii}


% % % % % % % % % % % % % %
% Section 1               %
% % % % % % % % % % % % % %
\section{Introduction}
\subsection{Identification}

The software system being considered for development is referred to as \textit{Freedom in the Star System}.  The customer providing specifications for the system is Dr. Clinton Jeffrey, University of Idaho.  The ultimate customer, or end-user, of the system will be the general public.  This is a new project effort, so the version under development is version 0.1.0.

\subsection{Purpose}
The purpose of the system under development is entertainment.  While the system will be used by the general public, this document is intended to be read and understood by UICS software
designers and coders. The document will also be vetted or approved by members of this team not involved in the writing of a given portion.

\subsection{Scope}
\textit{Freedom in the Galaxy} is a board game designed and owned by SPI (now Avalon Hill) in 1978.  \textit{Freedom in the Star System} will be a digital representation of the board game for the purposes of practicing software engineering standards in a controlled, educational environment.

\subsection{Definitions, Acronyms, and Abbreviations}

\begin{minipage}{\linewidth}
% Table
% Create column type for ``Brief Description''
\newcolumntype{P}{>{\centering\arraybackslash}p{\dimexpr.25\linewidth-2\tabcolsep}}
\centering
%\captionof*{table}{\textsc{\textbf{Record of Changes}}}
\begin{tabularx}{\textwidth}{lX}\toprule[1.5pt] % Description column is multiline

% Header
\bf Term of Acronym & \bf Definition\\ \midrule[1.0pt]

% Acronyms
Alpha test & Limited release(s) to inside testers and selected outside testers\\
Beta test & Limited release(s) to cooperating customers wanting early access to developing systems\\
Acceptance test & Final test; release of full functionality to customer for approval\\
SSRS & System and Software Requirements Specification; this document\\
SSDD & System and Software Design Document\\
FITG & Freedom in the Galaxy; board game developed by SPI/Avalon Hill\\
FITSS & Freedom in the Star System; partial digital adaptation of FITG\\
SPI & Simulations Publications Incorporated; original creators of FITG\\
GUI & Graphical User Interface; a visual client that a user can use to interact with the system\\
AI & Artificial Intelligence; a method of decision making that appears to make intelligent decisions to a given problem\\
LAN & Local Area Network; --\\

% Footer
\bottomrule[1.5pt]
\end{tabularx}
\end{minipage}


\subsection{References}
The following resources were used or referenced when creating this document:
\begin{itemize}
	\item SSRS \LaTeX\ document, Dr. Clinton Jeffey, \url{http://www2.cs.uidaho.edu/~jeffery/courses/383/templates/SSRS_Template_A2.tex}
	\item Sample SRS document: Geagea et. al., \url{http://www.cse.chalmers.se/~feldt/courses/reqeng/examples/srs_example_2010_group2.pdf}
\end{itemize}

\subsection{Overview and Restrictions}
This document is for limited release only to UI CS personnel working on
the project and those explicitly approved by UI CS personnel working on the project.

Section 2 of this document describes the system under development from a
holistic point of view. \ Functions, characteristics, constraints,
assumptions, dependencies, and overall requirements are defined from
the system-level perspective.

Section 3 of this document describes the specific requirements of the
system being developed. \ Interfaces, features, and specific
requirements are enumerated and described to a degree sufficient for a
knowledgeable designer or coder to begin crafting an architectural
solution to the proposed system.

Section 4 provides the requirements traceability information for the
project.  Each feature of the system is indexed by the SSRS
requirement number and linked to its SSDD and test references.

Sections 5 and up are appendices including original information and
communications used to create this document.


% % % % % % % % % % % % % %
% Section 2               %
% % % % % % % % % % % % % %

\section{Overall Description}

\subsection{Product Perspective}
\textit{Freedom in the Star System} will consist of three parts: a visual client that a human user will use to play an instance of a game, an AI that will serve as a non-human player and a server that will contain all of the game logic and state.  The server is further broken down into three subsystems: a system that handles game logic and rules, a database that will hold the current status of all games being played and a network interface that the visual client and AI will connect to in order to communicate with the backend.

\subsection{Product Functions}
Using the visual client (herein referred to as the GUI or the client), a user will be able to view a list of games and their options that are waiting for a second player to join in order to start.  A user will also be able to join one of these listed games in order to play a game against an opponent.  A user will be able to create a new game by selecting from a short list of preferences and joining the matchmaking queue.  Finally, a user will be able to create a new game against an AI (computer) opponent.  The server (also referred to as the backend) will be able to accept requests from either a human or non-human (AI) player and determine whether the received request is a valid request or not.  The server will then respond to the requesting user the results of their request and update any changes to the game state that a request may have affected.  The AI will act as a non-human player (or client in the network sense) that will attempt to send requests to the server, based on a specific strategy in order to attempt to win the game or prevent the opposing player from winning.

\subsection{User Characteristics}
There are primarily two different types of users that will use this system.  One user, the general player, is only interested in playing a game (also referred to as a match) with another player.  This type of user will try to avoid anything that relates to operating the server; the one exception to this being the user starting the server application in order to start the game logic engine.  The second, and less common user, is the user interested in running an instance of the server application for other users to connect to, essentially as a central host for other players to connect to.

\subsection{Constraints}
\textit{Freedom in the Star System} has very few constraints.  In order for two users to play a game together, the two computers must be connected to a common network.  This network can be a LAN, however most commonly this will be the Internet.  If a loss of the network connection occurs, the user will be able to restart their client and reconnect to their game.  If a loss of network connection happens where the server application resides, the server application will need to be re-executed - it will then retrieve its game state from the database and wait for new requests to be sent from the clients.  A user may use their computer's internal network to play a game against an AI opponent in which case a network connection loss won't create any issues (as long as the server application is being executed locally).  The only other constraints the \textit{Freedom in the Star System} application imposes is that it meets certain minimum system specifications as required by the 3rd party tools in use.

\subsection{Assumptions and Dependencies}
\textit{Freedom in the Star System} will operate under Linux, Mac OSX and Microsoft Windows.  All of the third party resources that are in use have support for these three operation systems as well.  \textit{Freedom in the Star System} depends on the following third party tools:
	\begin{itemize}
		\item Python 2.7.5 or higher
		\item The following Python Libraries
		\begin{itemize}
			\item PyGame
			\item RPyC
			\item SQLAlchemy
		\end{itemize}
	\end{itemize}

Earlier or later versions of the Python programming language may in fact work, but there is no guarantee that this is the case and Python 2.7.5 is the only officially supported version of Python.  All of the libraries that \textit{Freedom in the Star System} requires are supported by Python 2.7.5 at a minimum.

\subsection{System Level (Non-Functional) Requirements}
This subsection identifies system-level requirements that may impact the construction, operation, packaging and delivery of the system and software.

\subsubsection{Site Dependencies}
\textit{Freedom in the Star System} requires the following in order to run:
\begin{enumerate}
	\item Operating Systems
	\begin{enumerate}
		\item Linux 2.6 and above
		\item Mac OSX 10.7 and above
		\item Microsoft Windows XP and above
	\end{enumerate}
	\item Python 2.7.5
	\begin{enumerate}
		\item PyGame
		\item RPyC
		\item SQLAlchemy
	\end{enumerate}
	\item Broadband connection for Internet games
\end{enumerate}

\subsubsection{Safety, Security and Privacy Requirements}
There are no inherent safety, security or privacy requirements of this application.  As a precaution however, all users of the system should keep private those things that they would not want revealed in public; the user may be playing against another player and there is no way to control what that opponent may say and/or do given any private information.

\subsubsection{Performance Requirements}
%{\selectlanguage{english}\itshape\color{black}
%This paragraph should specify both the static and the dynamic numerical
%performance requirements placed on the soft ware or on human
%interaction as a whole. \ Static numerical requirements may include the
%following: a) The number of terminals to be supported; b) The number of
%simultaneous users to be supported; c) Amount and type of information
%to be handled. \ Dynamic numerical requirements may include, for
%example, the numbers of transactions and tasks and the amount of data
%to be processed within certain time periods for both normal and peak
%workload conditions. \ All of these requirements should be stated in
%measurable terms. For example, {\textquotedblleft}95\% of the
%transactions shall be processed in less than
%1msec.{\textquotedblright}}

The AI must be able to make decisions in under 90 seconds for any given hardware platform; after 90 seconds, if the best decision hasn't been found, then the best decision that has been found thus far should be selected in order to move on.

The Backend must be able to make most logic computations in under one second and not exceed three seconds in handling an entire request, from calculating the result to preparing and submitting the response.  The Backend has no control over network transfer speeds, so the actual time for a client to receive a response may be higher.

The Backend has no overall memory or hard disk space requirements because a single instance of the server could be hosting a large number of simultaneous games.  It is the responsibility of the server administrator to occasionally purge, backup or otherwise maintain the database created by the server.

\subsubsection{System and Software Quality}

\textit{Freedom in the Star System} should be:
\begin{itemize}
	\item Reliable
		\subitem The game should always produce the same outcomes given the same input and environment.
	\item Maintainable
		\subitem The source code should be easy to make changes to when needed.
	\item Portable
		\subitem The game should strive to remain available on Linux, Mac OSX and Windows.  This should include making sure that third party tools being used are also portable.
	\item Testable
		\subitem The game logic should be easily testable using unit tests.
		\subitem The GUI should be easily testable using a manual test battery.
		\subitem The AI logic should be easily testable using either unit tests, or barring that, a manual test battery.
\end{itemize}

\subsubsection{Packaging and Delivery Requirements}
%{\selectlanguage{english}\itshape\color{black}
%This paragraph shall specify the requirements, if any, for packaging,
%labeling, handling and delivery of the system being developed to the
%customer.}

The executable system and all associated documentation (i.e., SSRS, SSDD, Test Plan and Metrics Document and source code listing) will be available for the customer to acquire from the \textit{Freedom in the Star System} repository online at \url{https://github.com/ui-cs383/Freedom-Galaxy}.  Some associated documentation (i.e., SSRS, SSDD and Test Plan and Metrics Document) will also be delivered to the customer in hard copy.  Both of these deliverables will delivered, as specified by the customer, at the time of delivery.  Although document drops may occur throughout the system development process, the final, edited version of the above documents will accompany the final, accepted version of the executable system.

\subsubsection{Personnel-Related Requirements}
The system under development has no special personnel-related
characteristics.

\subsubsection{Training-related requirements}

No training materials or expectations are tied to this project other
than the limited help screens built into the software and the
accompanying user manual.

\subsubsection{Logistics-Related Requirements}
\textit{Freedom in the Star System} is supported under the following environments:
\begin{enumerate}
	\item Operating Systems
	\begin{enumerate}
		\item Linux 2.6 and above
		\item Mac OSX 10.7 and above
		\item Microsoft Windows XP and above
	\end{enumerate}
	\item Python 2.7.5
	\begin{enumerate}
		\item PyGame
		\item RPyC
		\item SQLAlchemy
	\end{enumerate}
	\item Broadband connection for Internet games
\end{enumerate}

\subsubsection{Other Requirements}
There are no additional requirements for the \textit{Freedom in the Star System} application that have not been explicitly stated in this document.

\subsubsection{Precedence and Criticality of Requirements}
%{\selectlanguage{english}\itshape\color{black}
%This paragraph shall specify, if applicable, the order of precedence,
%criticality, or assigned weights indicating the relative importance of
%the requirements in this specification. \ Examples include identifying
%those requirements deemed critical to safety, to security, or to
%privacy for purposes of singling them out for special treatment. \ If
%all requirements have equal weight, this paragraph shall so state. }

The only requirements that carry any greater weight as compared to the other requirements are the system and logistic requirements.  It is critical to the successful execution of \textit{Freedom in the Star System} to run \textit{Freedom in the Star System} on a supported operating system and for the requisite dependent libraries to be correctly installed.

%\clearpage
% % % % % % % % % % % % % %
% Section 3               %
% % % % % % % % % % % % % %
\section{Specific Requirements}
This section of the document will contain all of the software requirements in a sufficient level of detail to enable designers to design a system to satisfy those requirements and testers to test that the system satisfies those requirements.
%{\selectlanguage{english}\itshape\color{black}
%This section of the document should contain all of the software
%requirements to a level of detail sufficient to enable designers to
%design a system to satisfy those requirements, and testers to test that
%the system satisfies those requirements. \ Throughout this section,
%every stated requirement should be externally perceivable by users,
%operators, or other external systems. \ These requirements should
%include at a minimum a description of every input \ into the system,
%every output from the system, and all functions performed by the system
%in response to an input or in support of an output. \ As this is often
%the largest and most important part of the document, all requirements
%should be uniquely identifiable and careful attention should be given
%to organizing the requirements to maximize readability.}

\subsection{External Interface Requirements}
%{\selectlanguage{english}\itshape\color{black}
%This subsection should be a detailed description of all inputs into and
%outputs from the software system. \ It should complement the
%constraints and dependencies defined in earlier sections, but not
%repeat that information. \ Hardware, software, user, and other
%communication interfaces need to be specified. \ Use the four
%subsections listed below or the table on the next page, or some
%combination of both.}

\subsubsection{Hardware Interfaces}
No subsystem of \textit{Freedom in the Star System} relies on a hardware interface in order to operate.  The underlying operating systems controls any hardware that \textit{Freedom in the Star System} may depend on.

\subsubsection{Software Interfaces}
Some of the Python libraries that \textit{Freedom in the Star System} relies on will use different software (i.e., some of the Python Libraries will communicate with the operating system), \textit{Freedom in the Star System} itself has no software interfaces except to Python itself.

\subsubsection{User Interfaces}
The user interfaces can be separated by each subsystem.  The AI subsystem is the simplest - it has no user interface because it is executed by the Backend whenever it is needed.  The AI subsystem may produce a log file of actions occurred, but nothing else.  The Backend subsystem has a console based interface but the user cannot interact with it except to read the current status of the server and to terminate a currently running instance of the server.  The Client (GUI) subsystem however has a myriad of different user interfaces because that is what it has been designed for.  

The GUI will present two primary screens: an introduction screen, where the user will be able to create or join a match and the in-game screen which will contain all of the currently known information - information that the user is allowed to have knowledge of.  This information may include the status and quantity of one or both player's assets, the static objects of the game and information received about the current game state from the Backend.

The GUI will allow the user to send requests to change the game state to the Backend; the user may do this by using the ``drag and drop'' technique or by using keyboard and mouse hotkeys.  There are no plans currently to allow the GUI to function using only a keyboard, but using only a mouse will certainly be possible.

\subsubsection{Other Communication Interfaces}
The GUI and AI subsystems do not ever communicate with each other, but do in fact have to both communicate bi-directionally with the Backend subsystem.  In order to facilitate this, a network connection between the Backend and any client (human or AI) must be established.  \textit{Freedom in the Star System} uses RPyC in order to allow this.  The Backend and the Client may be on separate computers connected over the Internet, separate computers connected to an internal network or even running on the same computer using the loopback interface (commonly referred to as ``localhost'').  The AI subsystem, which is executed at the server's command still requires a network connection to be active and open for the duration of the game (it will likely run on the same machine as the server is running and will connect through localhost).


%\clearpage
%\setcounter{page}{1}\pagestyle{Convertiv}

\begin{minipage}{\linewidth}
\centering
\textbf{External Interface Requirements}
\end{minipage}

\begin{comment}
\begin{minipage}{\linewidth}
\centering
\captionof*{table}{\textbf{Hardware Interfaces}}
\begin{tabular}{c c c c c c } \toprule[1.5pt]
\bf Name & \bf Source/Destination & \bf Description & \bf Type/Range & \bf Dependencies & \bf Formats\\ \midrule[1.0pt]
 text & text & text & text & text & text \\
\bottomrule[1.5pt]
\end {tabular} %\par
\end{minipage}

\bigskip
\end{comment}

\begin{comment}
\begin{minipage}{\linewidth}
\centering
\captionof*{table}{\textbf{Software Interfaces}}
\begin{tabular}{c c c c c c } \toprule[1.5pt]
\bf Name & \bf Source/Destination & \bf Description & \bf Type/Range & \bf Dependencies & \bf Formats\\ \toprule[1.0pt]
--- & --- & --- & --- & --- & ---\\
--- & --- & --- & --- & --- & ---\\
--- & --- & --- & --- & --- & ---\\
\bottomrule[1.5pt]
\end {tabular} %\par
\end{minipage}

\bigskip
\end{comment}

\begin{minipage}{\linewidth}
\centering
\captionof*{table}{\textbf{User Interfaces}}
\begin{tabular}{c c c c c c } \toprule[1.5pt]
\bf Name & \bf Source/Destination & \bf Description & \bf Type/Range & \bf Dependencies & \bf Formats\\ \toprule[1.0pt]
Intro Screen & ---/Game Screen & Introduction screen for the game & --- & PyGame & --- \\
Game Screen & Intro Screen/--- & Main game screen for all activity & --- & PyGame & --- \\
\bottomrule[1.5pt]
\end {tabular} %\par
\end{minipage}

\bigskip

\begin{minipage}{\linewidth}
\centering
\captionof*{table}{\textbf{Other Communication Interfaces}}
\begin{tabular}{c c c c c c } \toprule[1.5pt]
\bf Name & \bf Source/Destination & \bf Description & \bf Type/Range & \bf Dependencies & \bf Formats\\ \toprule[1.0pt]
Connect Client & Client/Backend & Client connects to server & --- & RPyC & TCP/IP\\
Connect AI & AI/Backend & AI connects to server & --- & RPyC & TCP/IP\\
\bottomrule[1.5pt]
\end {tabular}\par
\end{minipage}

\clearpage %\setcounter{page}{1}\pagestyle{Convertv}
\subsection{System Features}
%{\selectlanguage{english}\itshape\color{black}
%Functional requirements should define the fundamental actions (i.e.,
%features) \ that must take place in the software in accepting and
%processing the inputs and in processing and generating the outputs.
%These requirements are given in the form of \textbf{Use Cases} where
%possible, denoting a concrete use (discrete user-performable task) of
%the system. Use case diagrams are followed by use case descriptions,
%followed by any non-task features. Non-task features are generally
%listed as {\textquotedblleft}shall{\textquotedblright} statements
%starting with {\textquotedblleft}The system
%shall{\dots}{\textquotedblright} \ These include: a) Validity checks on
%the inputs; b) Exact sequence of operations; c) Responses to abnormal
%situations, including error detection, handling and recovery; d)
%Parameter specification and usage; e) Relationship of outputs to
%inputs, including formulas for input to output conversion. \newline
%\newline
%It may be appropriate to partition the functional requirements into sub
%functions or subprocesses, but that decomposition (here) does not imply
%that the software design will also be partitioned that way. \ You
%should repeat subsections 3.2.i for every specified feature defined for
%the system or software.}

\subsubsection{Use Case Diagrams}
[insert 1+ use case diagrams here]

\clearpage
\subsubsection{System Feature 1: Setting up the Player(s)}
\begin{usecase}
  \addtitle{Pre-Game I}{Setting up the Player(s)}
  \addfield{Summary}{The players connect with each other.}
  \additemizedfield{Actors}{
    \item Human player(s)
  }
  \additemizedfield{Preconditions}{
    \item Game client(s) and server(s) have been started
  }
  \additemizedfield{Primary Scenario: Multi-player}{
    \item Players adjust settings and ensure general connectivity
    \item Players find each other either via server matching or other communication
    \item Players connect to each other for the game-start
    \item Players choose a side, one of:
      \begin{enumerate}
        \item Rebel Player
        \item Imperial Player
      \end{enumerate}
  }
  \additemizedfield{Alternative Scenario: Single-player}{
    \item Player adjusts settings
    \item Player chooses side, one of:
      \begin{enumerate}
        \item Rebel Player
        \item Imperial Player
      \end{enumerate}
    \item Player chooses AI difficulty level
  }
  \finalize
\end{usecase}

\clearpage
\subsubsection{System Feature 2: Choosing the Scenario}
\begin{usecase}
  \addtitle{Pre-game II}{Choosing the Scenario}
  \addfield{Summary}{The players choose from among the
    available scenarios for their chosen game.}
  \additemizedfield{Actors}{
    \item Imperial Player
    \item Rebel Player
  }
  \additemizedfield{Preconditions}{
    \item Game type has been chosen
  }
  \additemizedfield{Primary Scenario: Star-System Game}{
    \item The system displays the Star-System game scenarios and their information
          \begin{enumerate}
                \item Flight to Egrix
                \item The Varu Powderkeg
          \end{enumerate}
    \item The players vote to agreement on which scenario to play
  }
  \additemizedfield{Alternative Scenario: Province Game}{
    \item The system displays the Province game scenarios and their information
        \begin{enumerate}
              \item The Empire's Backdoor
              \item Orlog Besieged
        \end{enumerate}
    \item The players vote to agreement on which scenario to play
  }
  \additemizedfield{Alternative Scenario: Galactic Game}{
    \item There is only one scenario for the Galactic Game, and that is the Campaign
    \item The players
  }
    \finalize
\end{usecase}
\clearpage
\subsubsection{System Feature 3: Game Setup}
\begin{usecase}
	\addtitle{Game Setup I}{Game Setup}
	\addfield{Summary}{System sets up game state and then Players make purchases and place armies before the start of the first Game-Turn}
	\additemizedfield{Actors}{
		\item Rebel Player
		\item Imperial Player
	}
	\additemizedfield{Preconditions}{
		\item Pre-Game Setup has been completed
		}
	\addscenario{Primary Scenario}{
		\item System performs initial setup
		\item Imperial Player purchases armies
		\item Imperial Player places armies
		\item Rebel Player determines location of Rebel Secret Base (Galactic only)
	}
	\finalize
\end{usecase}

\clearpage
\subsubsection{System Feature 4: Environ Movement}
\begin{usecase}
  \addtitle{Movement I}{Environ Movement}
  \addfield{Summary}{Player moves Character(s) or military units from Environ to Environ on single planet}
  \additemizedfield{Actors}{
    \item Phasing Player
    \item Non-Phasing Player
  }
  \additemizedfield{Preconditions}{
    \item It is the movement segment of the phasing players turn or it is the reaction segment of the non-phasing player
  }
  \additemizedfield{Primary Scenario: Units not detectable by PDB}{
    \item Player chooses which units are to be moved and where to move them
    \item System moves units to new environ
  }
  \additemizedfield{Alternative Scenario: Units are dectectable by PDB}{
    \item System handles PDB Rules
  }
    \finalize
\end{usecase}

\clearpage
\subsubsection{System Feature 5: Interplanetary Movement}
\begin{usecase}
  \addtitle{Movement II}{Interplanetary Movement}
  \addfield{Summary}{Player moves eligible units between planets}
  \additemizedfield{Actors}{
        \item Phasing Player
        \item Non-Phasing Player
  }
  \additemizedfield{Preconditions}{
       \item Military units must be mobile or stacked with mobile military units. Characters must be on a ship
       \item non-phasing player may intercept
  }
  \additemizedfield{Primary Scenario: Travel occurs in single star system}{
                    \item Player chooses which units are to be moved and where to move them within the star system.
                    \item System processes PDB check.
                    \item Non Phasing player may react on per orbit basis, intercepting.
  }
  \additemizedfield{Alternative Scenario: Travel occurs between multiple star systems }{
                        \item Player chooses which units are to be moved inside one star system and the destination inside another star system.
                    \item System processes PDB check
                    \item Player Hyper Jumps
                    \item Non Phasing Player may react on per orbit basis
  }
    \finalize
\end{usecase}

\clearpage
\subsubsection{System Feature 6: Military Combat}
\begin{usecase}
  \addtitle{Military Combat I}{Military Combat}
  \addfield{Summary}{
  	Both players have military units in the same environ and the Movement and Enemy reaction segments of the operation phase are over. One of the players must decide to attack and the combat is resolved by the computer.
  	}
  \additemizedfield{Actors}{
  \item Phasing Player
  \item Non-Phasing Player
  }
  \additemizedfield{Preconditions}{
      \item Movement Segment of the Operations phase has been completed by the Phasing Player.
      \item Enemy Reaction Segment of the Operations phase has been completed by the Non-Phasing Player.
      \item There must be an environ with both players military units.
  }
  \addscenario{Primary Scenario}{
    \item Phasing Player chooses which environ to resolve conflict.
    \item Phasing Player decides if they want to attack.
    \item If yes...
      \begin{itemize}
        \item Players choose leaders if they can.
        \item The computer determines the outcome of the attack.
      \end{itemize}
    \item If no...
      \begin{itemize}
        \item Does Non-Phasing Player wish to attack?
        \item If yes...
          \begin{itemize}
            \item Players choose leaders if they can.
            \item The computer determines the outcome of the attack.
          \end{itemize}
        \item If not
        \item Phasing Player chooses next environ. Process repeats until there are no remaining conflicts.
        \end{itemize}
  	}
  	  \finalize
\end{usecase}

\clearpage
\subsubsection{System Feature 7: Military Combat after Rebellion}
\begin{usecase}
  \addtitle{Military Combat II}{Military Combat after Rebellion}
  \addfield{Summary}{A rebellion has occured and the Rebel player receives military units as a result.
                     Those rebel military units are also placed on an environ with imperial units. The rebel
                   player can choose to attack.}
  \additemizedfield{Actors}{
      \item Rebel Player
      \item Imperial Player
  }
  \additemizedfield{Preconditions}{
      \item A rebellion must have just occurred.
  }
  \addscenario{Primary Scenario}{
  \item Does the rebel wish to attack?
  \item If yes...
    \begin{itemize}
      \item Players choose leaders.
      \item Computer handles the attack.
    \end{itemize}
  \item If no...
  \item Nothing happens. Player Turn resumes.
  }
    \finalize
\end{usecase}


\clearpage
\subsubsection{System Feature 8: Character Versus Character Combat}
\begin{usecase}
	\addtitle{Character Combat I}{Phasing vs Non-Phasing}
	\addfield{Summary}{Non-Phasing Player's characters or military units have found Phasing Player characters and initiated combat, which shall be resolved immediately in a series of rounds of character combat procedure.}
	\additemizedfield{Actors}{
		\item Phasing Player
		\item Non-Phasing Player
	}
	\additemizedfield{Preconditions}{
		\item Non-Phasing Player's search has found enemy characters and has initiated combat.
		}
	\addscenario{Primary Scenario}{
			\item Determine whether the character combat will be \textit{hand to hand} or a \textit{firefight}.
			\begin{enumerate}
			\item If a squad is the attacking force, the combat is a firefight
			\item If characters are the attacking force, the Non-Phasing Player may dictate whether \textit{hand to hand} or a \textit{firefight}.
			\end{enumerate}
		\item Non-Phasing Player announces whether attempting to kill or capture.
			\begin{enumerate}
			\item If attempting to capture, shift two-columns to the left on the Character Combat Results Table for all resulting combat, including \textit{break off} attemps.
			\end{enumerate}
		\item Phasing Player assigns \textit{active} defenders from those available with at least one character required.
		\item System determines \textit{combat differential} and informs both Phasing and Non-Phasing players.
		\item Phasing Player may attempt to \textit{break off} from combat. 
			\begin{enumerate}
			\item Phasing Player attempts \textit{break off}
				\begin{enumerate}
				\item If \textit{break off} successful for only inactive defenders, those defenders are no longer found or involved in combat.
				\item If \textit{break off} successful for active AND inactive defenders, both are no longer found and combat ends.
				\item If \textit{break off} is unsuccessful defenders suffer a right column shift on the Character Combat Results Table for combat.
				\end{enumerate}
			\end{enumerate}
		\item Unless all defending characters have succeeded in breaking off, System determines combat consequences, i.e. number of wounds and captured characters.
		\item Both Phasing and Non-Phasing Player allocate respective wounds.
		\item Repeat steps 3-7 indefinitely as long as both sides have live uncaptured characters and defender has not broken off.
	}
	\addscenario{Alternative Scenario: Phasing Player Fully Breaks Off}{
		\item If Phasing Player \textit{breaks off} with all remaining units, combat ends.
	}
	\finalize
\end{usecase}

\clearpage
\subsubsection{System Feature 9: Character Versus Non-Player Combat}
\begin{usecase}
	\addtitle{Character Combat II}{Phasing vs Non-Player Attackers}
	\addfield{Summary}{Characters have encountered either a Creature or Irate Locals as an action event.}
	\additemizedfield{Actors}{
		\item Phasing Player
	}
	\additemizedfield{Preconditions}{
		\item Encountered a Creature or Irate Locals as an action event.
		}
	\addscenario{Primary Scenario}{
			\item Determine whether the character combat will be \textit{hand to hand} or a \textit{firefight}.
			\begin{enumerate}
			\item If a Creature is the attacking force, the combat is a \textit{hand to hand}
			\item If Irate Locals are the attacking force, reference the Irate Locals Chart to determine combat type.
			\end{enumerate}
		\item System informs Phase Player that attacker is attempting to kill, as Creatures and Irate Locals always do.
		\item Phasing Player assigns \textit{active} defenders from those available with at least one character required.
		\item System determines \textit{combat differential} and informs both Phasing and Non-Phasing players.
		\item Phasing Player may attempt to \textit{break off} from combat. 
			\begin{enumerate}
			\item Phasing Player attempts \textit{break off}
				\begin{enumerate}
				\item If \textit{break off} successful for only inactive defenders, those defenders are no longer found or involved in combat.
				\item If \textit{break off} successful for active AND inactive defenders, both are no longer found and combat ends.
				\item If \textit{break off} is unsuccessful defenders suffer a right column shift on the Character Combat Results Table for combat.
				\end{enumerate}
			\end{enumerate}
		\item Unless all defending characters have succeeded in breaking off, System determines combat consequences, i.e. number of wounds and captured characters.
		\item Phasing Player allocates wounds.
		\item System allocates wounds to Creature or Irate Locals.
		\item Repeat steps 3-7 indefinitely as long as both sides have live uncaptured characters and defender has not broken off.
	}
	\addscenario{Alternative Scenario: Phasing Player Fully Breaks Off}{
		\item If Phasing Player \textit{breaks off} with all remaining units, combat ends.
	}
	\additemizedfield{Related Use Cases}{
		\item Missions: Mission Phase
	}
	  \finalize
\end{usecase}

\clearpage
\subsubsection{System Feature 10: Search Phase}
\begin{usecase}
	\addtitle{Search I}{Search Phase}
	\addfield{Summary}{The Non-Phasing Player may attempt to search out detected enemy character(s). If the search is successful, the Non-Phasing Player gains intel on the enemy character(s) and may then engage in Character Combat. Searches are attempted with friendly character(s) and/or military units on a per-Environ basis where there are enemy character(s) detected.}
	\additemizedfield{Actors}{
	        \item Non-Phasing Player
	}
	\additemizedfield{Preconditions}{
	        \item Operations Phase has been completed and/or Phasing Player is attempting a mission.
	        \item Non-Phasing player has character(s) and/or military units present in an Environ with detected enemy character(s).
	        }
	\addscenario{Primary Scenario}{                                         
	        \item                   
	                \begin{enumerate}
	                        \item Non-Phasing Player chooses an eligible Environ.                                                   \begin{center}
	                        ? -- OR --
	                        \end{center}
	                        \item Phasing Player has character(s) attempting a mission and has become detected.
	                \end{enumerate}
	        \item System determines and displays eligible characters and/or military units.
	        \item Non-Phasing Player chooses eligible units to attempt a search.
	                
	        \item If the search is successful, the Non-Phasing player may decide to engage the enemy in Character Combat.
	        \item Non-Phasing Player may choose the next eligible Environ to perform a search in and the process repeats until no eligible Environs remain.
	                \begin{enumerate}
	                        \item If the Phasing Player has character(s) attempting a mission and becomes detected, steps 2-4 are repeated.
	                \end{enumerate}
	}
	\addscenario{Alternative Scenario: Do Not Search}{
	        \item Non-Phasing Player may choose not to search.  In this case, the Search Phase is complete.
	}
	\addscenario{Alternative Scenario: Search Not Possible}{
	        \item No eligible Environs are present for searching; the Search phase is skipped.
	}
	\additemizedfield{Related Use Cases}{
	        \item Character Combat: %waiting for character combat use case
	}
	  \finalize
\end{usecase}

\clearpage
\subsubsection{System Feature 11: Using Possessions}
\begin{usecase}
    \addtitle{Possessions I}{Using Possessions}
    \addfield{Summary}{The Rebel Player may attempt to use Possession(s) owned by his/her character(s) during an eligible phase. Once the item is used, the system determines the state of the Possession.}
    \additemizedfield{Actors}{
            \item Rebel Player 
    }
    \additemizedfield{Preconditions}{
            \item Rebel player is in an eligible phase to use a Possession in their control.
            }
    \addscenario{Primary Scenario}{                                                     
                                     \item Rebel player opens their inventory to look at their available Possessions.                        
            \item Rebel player chooses a Possession they wish to use.
            \item The item is used and its state is determined by the system.
            \item Rebel player may continue to use available Possessions.
            \item Rebel player closes inventory.
            \item The phase continues.
    }
    \addscenario{Alternative Scenario: Possessions Not Used}{
            \item Rebel player doesn't use a Possession and closes inventory.
    }
    \additemizedfield{Related Use Cases}{
            \item None
    }
      \finalize
\end{usecase}

\clearpage
\subsubsection{System Feature 12: Transferring Possessions}
\begin{usecase}
    \addtitle{Possessions II}{Transfer Possessions}
    \addfield{Summary}{The Rebel Player may assign a Possession to another character that is stacked with the character who currently owns the Possession.}
    \additemizedfield{Actors}{
            \item Rebel Player
    }
    \additemizedfield{Preconditions}{
            \item Rebel player is beginning the Mission Phase and has stacked characters with Possessions.
            }
    \addscenario{Primary Scenario}{                                                     
            \item Rebel player opens their inventory to look at their available Possessions.                        
            \item Rebel player chooses a Possession they wish to reassign.
            \item Rebel player chooses another character the Possession is reassigned to.
            \item Rebel player closes inventory.
            \item The phase continues.
    }
    \addscenario{Alternative Scenario: Possessions Not Transfered}{
            \item Rebel player doesn't transfer a Possession and closes inventory.
    }
    \additemizedfield{Related Use Cases}{
            \item None
    }
      \finalize
\end{usecase}

\clearpage
\subsubsection{System Feature 13: Missions}
\begin{usecase}
	\addtitle{Missions I}{Completing Missions in One Environ}
	\addfield{Summary}{The Phasing Player attempts to complete Missions by completing the Events chosen by the System.
	}
	\additemizedfield{Actors}{
		\item Phasing Player (regularly)
		\item Non-Phasing Player (occasionally)
	}
	\additemizedfield{Preconditions}{
		\item Mission Groups have been chosen, and Missions have been Assigned.
	}
	\addscenario{Primary Scenario}{
		\item System determines number of ``Mission Attempts'' and determines Action Events for each Attempt.
		\item Phasing Player completes Action Event.
			%\begin{enumerate}
			%	\item Non-Phasing Player may be involved with an Action Event.
			%\end{enumerate}
		\item Missions are completed based on results of Action Event.
		\item Phasing Player continues attempting Missions until:
			\begin{enumerate}
			\item All Missions are completed (successfully or unsuccessfully) or Aborted.
			\item No Mission Attempts remain.
			\end{enumerate}
		\item System calculates results of Bonus Mission Attempts for incomplete missions.
	}
	  \finalize
\end{usecase}



% % % % % % % % % % % % % %
% Section 4               %
% % % % % % % % % % % % % %
\section{Requirements Traceability}
%{\selectlanguage{english}\itshape\color{black}
%This section shall contain traceability information from each system
%requirement in this specification to the system (or subsystem, if
%applicable) requirements it addresses. \ A tabular form is preferred,
%but not mandatory.}

\subsection{Alpha Release}
\begin{minipage}{\linewidth}
% Table
\centering
\captionof*{table}{\textbf{Alpha Release}}
\begin{tabularx}{\textwidth}{*{2}{c}X*{4}{c}}\toprule[1.5pt] % Description column is multiline

% Header
\bf Feature Name & \bf Req. No. & \bf Req. \newline Description & \bf Priority & \bf SDD & \bf Test Case(s) & \bf Test Result(s) \\ \midrule[1.0pt]

% Features
Feature A & 1.1 & It does something over and over and over & M & 2.2 & 2 & 75\% \\

% Footer
\bottomrule[1.5pt]
\end{tabularx}\par

% Legend
\bigskip
\raggedleft
\begin{tabular}{c l} %{.3\textwidth}{c L}
\multicolumn{2}{c}{\textsc{Legend}} \\ \midrule[0.5pt]
\textsc{\textbf{L}}   & Low Priority\\
\textsc{\textbf{H}}   & High Priority\\
\textsc{\textbf{M}}	  & Mandatory Priority\\
\textsc{\textbf{P}}   & Passed Test Case\\
\textsc{\textbf{F}}   & Failed Test Case\\
\textsc{\textbf{SDD}} & \parbox{3cm}{\vspace{.25em}
						Link is version\\
						and page number\\
						or function name}
\end{tabular}
\end{minipage}

\begin{comment}
\subsection{Beta Release}
\begin{minipage}{\linewidth}
% Table
\centering
\captionof*{table}{\textbf{Beta Release}}
\begin{tabularx}{\textwidth}{*{2}{c}X*{4}{c}}\toprule[1.5pt] % Description column is multiline

% Header
\bf Feature Name & \bf Req. No. & \bf Req. \newline Description & \bf Priority & \bf SDD & \bf Test Case(s) & \bf Test Result(s) \\ \midrule[1.0pt]

% Features
Feature A & 1.1 & It does something over and over and over & M & 2.2 & 2 & 75\% \\

% Footer
\bottomrule[1.5pt]
\end{tabularx}\par

% Legend
\bigskip
\raggedleft
\begin{tabular}{c l} %{.3\textwidth}{c L}
\multicolumn{2}{c}{\textsc{Legend}} \\ \midrule[0.5pt]
\textsc{\textbf{L}}   & Low Priority\\
\textsc{\textbf{H}}   & High Priority\\
\textsc{\textbf{M}}	  & Mandatory Priority\\
\textsc{\textbf{P}}   & Passed Test Case\\
\textsc{\textbf{F}}   & Failed Test Case\\
\textsc{\textbf{SDD}} & \parbox{3cm}{\vspace{.25em}
						Link is version\\
						and page number\\
						or function name}
\end{tabular}
\end{minipage}
\end{comment}

\begin{comment}
\subsection{Accepted Release}
\begin{minipage}{\linewidth}
% Table
\centering
\captionof*{table}{\textbf{Accepted Release}}
\begin{tabularx}{\textwidth}{*{2}{c}X*{4}{c}}\toprule[1.5pt] % Description column is multiline

% Header
\bf Feature Name & \bf Req. No. & \bf Req. \newline Description & \bf Priority & \bf SDD & \bf Test Case(s) & \bf Test Result(s) \\ \midrule[1.0pt]

% Features
Feature A & 1.1 & It does something over and over and over & M & 2.2 & 2 & 75\% \\

% Footer
\bottomrule[1.5pt]
\end{tabularx}\par

% Legend
\bigskip
\raggedleft
\begin{tabular}{c l} %{.3\textwidth}{c L}
\multicolumn{2}{c}{\textsc{Legend}} \\ \midrule[0.5pt]
\textsc{\textbf{L}}   & Low Priority\\
\textsc{\textbf{H}}   & High Priority\\
\textsc{\textbf{M}}	  & Mandatory Priority\\
\textsc{\textbf{P}}   & Passed Test Case\\
\textsc{\textbf{F}}   & Failed Test Case\\
\textsc{\textbf{SDD}} & \parbox{3cm}{\vspace{.25em}
						Link is version\\
						and page number\\
						or function name}
\end{tabular}
\end{minipage}
\end{comment}

\end{document}
